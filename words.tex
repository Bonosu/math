\documentclass[11pt,a4paper]{jreport}
%
\usepackage{amsmath,amssymb}
\usepackage{bm}
\usepackage{graphicx}
\usepackage{ascmac}
\usepackage{float}
%
\setlength{\textheight}{40\baselineskip}
\addtolength{\textheight}{\topskip}
\setlength{\voffset}{-0.2in}
\setlength{\topmargin}{0pt}
\setlength{\headheight}{0pt}
\setlength{\headsep}{0pt}

\setlength{\textwidth}{\paperwidth}     % ひとまず紙面を本文領域に
\setlength{\oddsidemargin}{-5.4truemm}  % 左の余白を20mm(=1inch-5.4mm)に
\setlength{\evensidemargin}{-5.4truemm} % 
\addtolength{\textwidth}{-40truemm}     % 右の余白も20mmに
%
\newcommand{\divergence}{\mathrm{div}\,}  %ダイバージェンス
\newcommand{\grad}{\mathrm{grad}\,}  %グラディエント
\newcommand{\rot}{\mathrm{rot}\,}  %ローテーション
%
\title{脳を学ぶ上で重要な数学シリーズ 定義編}
\author{後藤 優仁}
\date{\today}
\begin{document}
\maketitle
%
%
\tableofcontents

\chapter{分類わからん}
\section{関数}
ここでは,ある変数に依存して決まる値,あるいはその対応関係の事を指す.すなわち古典的な関数の定義.集合論的なあれは扱わない.わからん.
\section{時間関数}
時間発展する関数.言い換えるなら横軸(どこかの独立した軸)に時間を用いてるもの.英語ではDynamics(ダイナミクス)などと称したりする.脳波は自明にこれ.時間と電位のダイナミクスですね.逆にダイナミクスじゃないのは,例えば国語の成績と数学の成績の関数など.
\subsection{非線形ダイナミクス}
ちなみに,脳波はダイナミクスのうちでも非線形ダイナミクスの分類に入る.非線形の説明は後に譲るが,簡単に言うと分離可能な独立した系である事である.脳波は分離する事が難しく,様々な要因によって値が変動する時間関数であるため,非線形なダイナミクスの一種である.カオス理論は聞き覚えがある事だろう.あれはまさに非線形ダイナミクスの話.

\chapter{代数学?}
\section{空間}
ここでは,ある集合とその集合内における要素同士の関係性が定義された状態をさすという事にしとく.
\subsection{ユークリッド空間}
んーと,とりあえず僕らの知ってる空間.1と2の間も,10000と10001の間も同じ長さだったりとか,そういう直感的な空間.
\subsection{非ユークリッド空間}
例えば地球儀とか,縮尺がなんか歪んでたりするキモイ空間.普通に脳の数学してる分には触らない.ただし,我々の脳は非ユークリッド空間を処理していたりするので知っとくと良い事あるかも.(網膜は二次元なので,光の密度的に視界中央部と周辺部では縮尺が異なったりするはずだけど,視空間はそんな事がなく距離の恒常性がたもたれているね?)
\subsection{ユークリッド平面}
いわゆる,x軸とy軸によって定義される平面の事.中学で習う関数のあれ.軸として利用されるのは,独立した実数.たとえば長さと重さ.数学的な直交性ではなく,表現が難しいため一般に座標を用いて(x,y)などのように表す.そこから,直交座標系などとも表現される.
\subsection{ガウス平面}
ユークリッドに代わり,平面として導入されたもの.軸には実数と虚数という,数学的に直交したものが使用されているため,平面上の一点を足し算の形で表す事で,一つの数として捉える事が可能になっている.虚数成分のiは絶対値,ノルムが1で実数の1と一致するため,距離のバランスが取れている.そのため,ユークリッド平面にも変換する事が可能であり,(x,y) = x + iyとなる.便利.
\subsection{極座標系}
直交座標系の点を異なる表現で表す系.当然縮尺や数の対応は保たれている.原点からの距離(ノルム r)と始線(横軸)との偏角($\theta$)の組で表現する.即ち,直交座標系での(1,1) は ($\sqrt{2}, \frac{\pi}{4}$)となる.

\section{内積}
\subsection{内積}
いやほんと,こいつの定義が一番ややこしいんだが.ベクトルにおいては「似ている度」を表すのは本編で確認した通り.ここをもう少し踏み込むと,ベクトルaをベクトルbに対して射影(上から直角に照らす)したときに出来る影の長さと捉えると良いです.二つのベクトルがほぼ同じ方向を向いてるのであれば,どちらかを他方に対し垂直に射影しても長さはほぼ同じままですが,直角に交わっていたら出来る影は0ですね?
\subsection{外積}
本編ではそっぽ向いてる度と言ったが,より正確には,まず元の2つのベクトルに対し直角(第3軸と考えるといいかも)方向に,元のベクトル2つによって定義される平行四辺形の面積と等しい長さのベクトルです.脳波の話では使わない...かなぁ?使うかも.
\subsection{点乗積}
\section{絶対値}
\section{直交性,線形独立性}
\section{射影}


\chapter{幾何学?}
\section{周期関数}
\subsection{位相}
\subsection{周波数と角周波数}
\section{偏角}

\chapter{解析学?}
\section{離散値と連続値}
\subsection{極限,連続性}
\section{フーリエ変換}
\subsection{フーリエ級数展開}
\subsection{フーリエ変換}
\section{ウェーブレット変換}
\subsection{マザーウェーブレット}
\section{ヒルベルト変換}
\section{活動源推定}
\subsection{逆問題}

\chapter{統計学?}
\section{統計量}


\end{document}