\documentclass[11pt,a4paper]{jreport}
%
\usepackage{amsmath,amssymb}
\usepackage{bm}
\usepackage{graphicx}
\usepackage{ascmac}
\usepackage{float}
%
\setlength{\textheight}{40\baselineskip}
\addtolength{\textheight}{\topskip}
\setlength{\voffset}{-0.2in}
\setlength{\topmargin}{0pt}
\setlength{\headheight}{0pt}
\setlength{\headsep}{0pt}

\setlength{\textwidth}{\paperwidth}     % ひとまず紙面を本文領域に
\setlength{\oddsidemargin}{-5.4truemm}  % 左の余白を20mm(=1inch-5.4mm)に
\setlength{\evensidemargin}{-5.4truemm} % 
\addtolength{\textwidth}{-40truemm}     % 右の余白も20mmに
%
\newcommand{\divergence}{\mathrm{div}\,}  %ダイバージェンス
\newcommand{\grad}{\mathrm{grad}\,}  %グラディエント
\newcommand{\rot}{\mathrm{rot}\,}  %ローテーション
%
\title{脳を学ぶ上で重要な数学シリーズ 定義編}
\author{後藤 優仁}
\date{\today}
\begin{document}
\maketitle
%
%
\tableofcontents

\chapter{分類わからん}
\section{関数}
ここでは,ある変数に依存して決まる値,あるいはその対応関係の事を指す.すなわち古典的な関数の定義.集合論的なあれは扱わない.わからん.
\section{時間関数}
時間発展する関数.言い換えるなら横軸(どこかの独立した軸)に時間を用いてるもの.英語ではDynamics(ダイナミクス)などと称したりする.脳波は自明にこれ.時間と電位のダイナミクスですね.逆にダイナミクスじゃないのは,例えば国語の成績と数学の成績の関数など.
\subsection{非線形ダイナミクス}
ちなみに,脳波はダイナミクスのうちでも非線形ダイナミクスの分類に入る.線形の説明は後に譲るが,簡単に言うと分離可能な独立した系である事である.脳波は分離する事が難しく,様々な要因によって値が変動する時間関数であるため,非線形なダイナミクスの一種である.カオス理論は聞き覚えがある事だろう.あれはまさに非線形ダイナミクスの話.

\chapter{代数学?}
\section{空間}
ここでは,ある集合とその集合内における要素同士の関係性が定義された状態をさすという事にしとく.
\subsection{ユークリッド空間}
んーと,とりあえず僕らの知ってる空間.1と2の間も,10000と10001の間も同じ長さだったりとか,そういう直感的な空間.
\subsection{非ユークリッド空間}
例えば地球儀とか,縮尺がなんか歪んでたりするキモイ空間.普通に脳の数学してる分には触らない.ただし,我々の脳は非ユークリッド空間を処理していたりするので知っとくと良い事あるかも.(網膜は二次元なので,光の密度的に視界中央部と周辺部では縮尺が異なったりするはずだけど,視空間はそんな事がなく距離の恒常性がたもたれているね?)
\subsection{ユークリッド平面}
いわゆる,x軸とy軸によって定義される平面の事.中学で習う関数のあれ.軸として利用されるのは,独立した実数.たとえば長さと重さ.数学的な直交性ではなく,表現が難しいため一般に座標を用いて(x,y)などのように表す.そこから,直交座標系などとも表現される.
\subsection{ガウス平面}
ユークリッドに代わり,平面として導入されたもの.軸には実数と虚数という,数学的に直交したものが使用されているため,平面上の一点を足し算の形で表す事で,一つの数として捉える事が可能になっている.虚数成分のiは絶対値,ノルムが1で実数の1と一致するため,距離のバランスが取れている.そのため,ユークリッド平面にも変換する事が可能であり,(x,y) = x + iyとなる.便利.
\subsection{極座標系}
直交座標系の点を異なる表現で表す系.当然縮尺や数の対応は保たれている.原点からの距離(ノルム r)と始線(横軸)との偏角($\theta$)の組で表現する.即ち,直交座標系での(1,1) は ($\sqrt{2}, \frac{\pi}{4}$)となる.

\section{内積}
\subsection{内積}
いやほんと,こいつの定義が一番ややこしいんだが.ベクトルにおいては「似ている度」を表すのは本編で確認した通り.ここをもう少し踏み込むと,ベクトルaをベクトルbに対して射影(上から直角に照らす)したときに出来る影の長さと捉えると良いです.二つのベクトルがほぼ同じ方向を向いてるのであれば,どちらかを他方に対し垂直に射影しても長さはほぼ同じままですが,直角に交わっていたら出来る影は0ですね?次元が増えてもこのノリで理解していけばいいような,よくないような...勉強が足りない.
\subsection{外積}
本編ではそっぽ向いてる度と言ったが,より正確には,まず元の2つのベクトルに対し直角(第3軸と考えるといいかも)方向に,元のベクトル2つによって定義される平行四辺形の面積と等しい長さのベクトルです.脳波の話では使わない...かなぁ?使うかも.
\subsection{点乗積}
\section{絶対値}
ユークリッドなりガウスなりの平面で考えるなら,原点から任意の点までの距離です.では距離の定義は何かという話になるわけだけど,面倒なのでピタゴラスの定理で求めるあれって事にしましょう.なので平面の系によって計算方法は若干変わるけど意味するところは同じ.
\section{直交性,線形独立性}
独立している事をさす.正確には違うけど,とりあえず全部同じって事で良いと思う.直交性があるなら線形独立だろうし.グラフを書いたときに軸として取り入れる事ができ,かつそれらが直交(二次元平面なら直角のイメージ)している状態.x軸とy軸.実数と虚数.明確に分離可能なもの.まあそんな感じ.
\section{射影}
影.ベクトルaのベクトルbに対する射影と言ったら,ベクトルaをbに対して垂直方向から光で照らす事.あるいは照らして,b軸方向に出来る影の事.内積のやつ.


\chapter{幾何学?}
\section{周期関数}
周期的に同じ波形が繰り返される関数.一次関数や二次関数は違うけど三角関数はそう.これ,結構大事な概念で,周期性がある場合は計算が楽になる(三角関数の時もそうだけど,積分が無限から無限じゃなくて一周期分で済んだね?).ので実際には非周期と思われる信号に対しても,時間窓を切って周期性を仮定して無理やり計算したりしている.実は脳波もそうしてる.というかフーリエやるにはそれしかない.周期の単位はs.

\subsection{位相}
位相とは,周期性を持つ現象において,その周期中の位置を表す量.無次元.基本は角度で表されて,一周期は$2\pi$なので0から$2\pi$のうちで表される事が多い.あるいは$-\pi$から$\pi$でも表す.意味は同じだね.
\begin{eqnarray}
y(t) = Asin(\omega t + a)
\end{eqnarray}
という三角関数で考えた場合,位相は$(\theta t + a)$であり,このうちaはt=0の時に残る項,つまり初期項,すなわち初期位相を表し,関数の初期位相位置である.またAは三角波の振幅である.\\
尚,オイラーの公式を用いて三角関数を指数関数に飛ばした場合,位相情報は指数部分に乗るため先程の式は

\begin{eqnarray}
y(t) = Ae^{(i\theta t + a)}
\end{eqnarray}
と翻訳される.例によってAは振幅であり,$\theta$は位相である.\\
\\
位相は無次元の角度であるため,加算が可能であり,これによって二つの波の位相差が定義可能であり,以下のようになる.

\begin{eqnarray}
diff(t) = (\theta_1 t +a) - (\theta_2 t + b) =e^{i((\theta_1 t+a) - (\theta_2 t+b))}
\end{eqnarray}
これが後に位相同期解析などに用いられたりする.
\subsection{周波数と角周波数}
周波数とは,一秒間に周期関数の周期が何回進むかであり,周波数が高い波はつまり一秒間に多くの周期が進む波,いわゆる速波とされる.逆もしかり.単位はHz.脳波として我々がよく用いるのは1Hzから40Hzくらいまで.もちろんこれより低い脳波も高い脳波も存在するが,諸々の事情で解析に用いるのが難しかったりする.つまりたとえば1000Hzの波を取りたいとして,これをちゃんと見るためには当然「1秒の間に1000回の波が通過した」事を観測する必要がある.一番大きな問題は,単純にそんな細かい精度で計測できない事.次に,もし計測できたとしても細かすぎて解析するときに情報が崩れちゃう.閑話休題.\\
角周波数は,周波数を角度で考えたもの.位相の考えに戻ると,一周期の間に位相は$2\pi$進むんだったね.これは定義からしてそういうもの.で,周波数ってのは1秒の間に何周期進んだかをさす.ので単純に,周波数に$2\pi$かけたものが角周波数.
\begin{eqnarray}
f = \frac{1}{T}\\
\omega = 2\pi f
\end{eqnarray}
ただしここでTは周期[s]で,fは周波数[Hz],$\omega$は角周波数[rad/s].
\section{偏角}
極座標系において点の位置を定める時に使われる量.始線(x軸てきな)と,任意の点と原点を結んだ線とのなす角.

\chapter{解析学?}
\section{離散値と連続値}
離散値は連続値じゃない値の事.連続値とは,ε-δ論法で定義づけられるあれ.つまり0と1の間には0.1があって, その間には0.01があってを無限に行えるのが連続値.離散値はできない.例えば人数は離散値.じゃあ逆に何なら連続なんだって言うと難しくて,我々が観測できるもの(認識,ではなく観測)には技術的限界とかがあるので結局離散値になっちゃうんだが,原理的には例えば温度の変化とかって連続的なもののはずだよね?そんな感じ.連続だと積分ができる.微分もできる.便利.
\subsection{極限,連続性}
上の定義から導かれると思う.上が分かれば,調べればわかる.
\section{フーリエ変換}
\subsection{フーリエ級数展開}
任意の関数を異なる周波数の三角関数の足し合わせの形に展開する事.
\subsection{フーリエ変換}
展開するだけじゃなく,そもそも周波数関数に変換する事.時間関数から周波数関数への変換.つまり時間情報を失う代わりに,周波数情報を取り出す.またこれを,短い時間窓にくぎって何度もやる事で疑似的に時間情報を復元する方法として短時間フーリエとか連続フーリエとか色々いるけど基本のフーリエさえわかればなんとかなる.
\section{ウェーブレット変換}
\subsection{マザーウェーブレット}
ウェーブレット変換において利用するカーネルの総称.いろんな形の関数があり,三角波にガウスをかけたやつが脳でよく使われてるけど,他にも色々いる.調べると面白いかもね.
\section{ヒルベルト変換}
フーリエやウェーブレットとは毛色が違うけど,目的は同じやつ.周波数情報を取り出せる.
\section{活動源推定}
\subsection{逆問題}
aがあるからbが起きるよねが順問題だけどbが起きたってことはaが起きた?が逆問題.脳の場合,こういう脳波が観測されたからここが活動してたはずて推定する.これが活動源推定.
\chapter{統計学?}
\section{統計量}
標本集団の性質を要約して説明してくれる量.

\end{document}